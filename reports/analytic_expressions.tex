% Options for packages loaded elsewhere
\PassOptionsToPackage{unicode}{hyperref}
\PassOptionsToPackage{hyphens}{url}
%
\documentclass[
]{article}
\usepackage{amsmath,amssymb}
\usepackage{lmodern}
\usepackage{iftex}
\ifPDFTeX
  \usepackage[T1]{fontenc}
  \usepackage[utf8]{inputenc}
  \usepackage{textcomp} % provide euro and other symbols
\else % if luatex or xetex
  \usepackage{unicode-math}
  \defaultfontfeatures{Scale=MatchLowercase}
  \defaultfontfeatures[\rmfamily]{Ligatures=TeX,Scale=1}
\fi
% Use upquote if available, for straight quotes in verbatim environments
\IfFileExists{upquote.sty}{\usepackage{upquote}}{}
\IfFileExists{microtype.sty}{% use microtype if available
  \usepackage[]{microtype}
  \UseMicrotypeSet[protrusion]{basicmath} % disable protrusion for tt fonts
}{}
\makeatletter
\@ifundefined{KOMAClassName}{% if non-KOMA class
  \IfFileExists{parskip.sty}{%
    \usepackage{parskip}
  }{% else
    \setlength{\parindent}{0pt}
    \setlength{\parskip}{6pt plus 2pt minus 1pt}}
}{% if KOMA class
  \KOMAoptions{parskip=half}}
\makeatother
\usepackage{xcolor}
\IfFileExists{xurl.sty}{\usepackage{xurl}}{} % add URL line breaks if available
\IfFileExists{bookmark.sty}{\usepackage{bookmark}}{\usepackage{hyperref}}
\hypersetup{
  pdftitle={DRAGON analytic expressions},
  pdfauthor={Kate},
  hidelinks,
  pdfcreator={LaTeX via pandoc}}
\urlstyle{same} % disable monospaced font for URLs
\usepackage[margin=1in]{geometry}
\usepackage{graphicx}
\makeatletter
\def\maxwidth{\ifdim\Gin@nat@width>\linewidth\linewidth\else\Gin@nat@width\fi}
\def\maxheight{\ifdim\Gin@nat@height>\textheight\textheight\else\Gin@nat@height\fi}
\makeatother
% Scale images if necessary, so that they will not overflow the page
% margins by default, and it is still possible to overwrite the defaults
% using explicit options in \includegraphics[width, height, ...]{}
\setkeys{Gin}{width=\maxwidth,height=\maxheight,keepaspectratio}
% Set default figure placement to htbp
\makeatletter
\def\fps@figure{htbp}
\makeatother
\setlength{\emergencystretch}{3em} % prevent overfull lines
\providecommand{\tightlist}{%
  \setlength{\itemsep}{0pt}\setlength{\parskip}{0pt}}
\setcounter{secnumdepth}{-\maxdimen} % remove section numbering
\ifLuaTeX
  \usepackage{selnolig}  % disable illegal ligatures
\fi

\title{DRAGON analytic expressions}
\author{Kate}
\date{2022-05-30}

\begin{document}
\maketitle

\hypertarget{goal}{%
\subsection{Goal}\label{goal}}

Equation 8 of the main manuscript decomposes the quantity

\[
R = E\left[||\hat\Sigma - \Sigma||^2_F\right]
\]

into terms which can be efficiently calculated analytically, yielding
estimates for optimal regularization parameters \(\lambda_1\) and
\(\lambda_2\) of DRAGON in the 2-omic case. Here, we derive this
equation.

\hypertarget{notation}{%
\subsection{Notation}\label{notation}}

Let \(\Sigma = \{\sigma_{ij}\}\) be the true population covariance and
let \(S = \{s_{ij}\}\) be sample covariance, i.e., the usual maximum
likelihood estimator for \(\Sigma\). Next, let
\(\hat\Sigma = \{\hat{\sigma}_{ij}\}\) be the covariance shrinkage
estimator defined in the main manuscript, i.e.:

\[
\hat\Sigma = \{\hat\sigma_{ij}\} = \left(\begin{matrix}
(1-\lambda_1)S^{(1,1)} & \sqrt{1-\lambda_1}\sqrt{1-\lambda_2}S^{(1,2)} \\
\sqrt{1-\lambda_2}\sqrt{1-\lambda_1}S^{(2,1)} & (1-\lambda_2)S^{(2,2)} \\
\end{matrix}
\right) + \left(\begin{matrix} 
\lambda_1 \text{diag}(S^{(1,1)}) & 0 \\ 
0 & \lambda_2 \text{diag}(S^{(2,2)})) \\
\end{matrix}\right)
\]

where we have divided \(S\) into blocks consisting of \(S^{(1,1)}\), the
sample covariance of omics layer 1; \(S^{(2,2)}\), the sample covariance
of omics layer 2; and \(S^{(1,2)} = t(S^{(2,1)})\), the sample
covariance between omics layer 1 and omics layer 2.

\hypertarget{derivation}{%
\subsection{Derivation}\label{derivation}}

We derive an expression for \(R\) in terms of \(\lambda_1\) and
\(\lambda_2\) by considering three cases:

\begin{enumerate}
\def\labelenumi{(\roman{enumi})}
\tightlist
\item
  \(i,j\) such that \(\sigma_{ij}\) represents the within-omic
  covariance of two variables \(i\) and \(j\), both in omics layer 1
\item
  \(i,j\) such that \(\sigma_{ij}\) represents the within-omic
  covariance of two variables \(i\) and \(j\), both in omics layer 2
\item
  \(i,j\) such that \(\sigma_{ij}\) represents the between-omic
  covariance of variable \(i\), which is in omics layer 1, and variable
  \(j\), which is in omics layer 2.
\end{enumerate}

In deriving these steps, we partition \(R\) into the sum of four
components: \(R^{(1,1)}\), obtained from \(\Sigma^{(1,1)}\) and
\(S^{(1,1)}\); \(R^{(2,2)}\), obtained from \(\Sigma^{(2,2)}\) and
\(S^{(2,2)}\); \(R^{(1,2)}\), obtained from \(\Sigma^{(1,2)}\) and
\(S^{(1,2)}\); and \(R^{(2,1)}\), obtained from \(\Sigma^{(2,1)}\) and
\(S^{(2,1)}\). From these, we can write the final equation

\[
\begin{aligned}
R &= R^{(1,1)} + R^{(2,2)} + R^{(1,2)} + R^{(2,1)} \\
&= R^{(1,1)} + R^{(2,2)} + 2R^{(1,2)}
\end{aligned}
\] where the last line arises due to the symmetric nature of the
covariance matrix and the Frobenius norm.

\hypertarget{case-i-both-variables-in-omics-layer-1}{%
\subsubsection{Case (i): both variables in omics layer
1}\label{case-i-both-variables-in-omics-layer-1}}

In this case, we have

\[
\begin{aligned}
R^{(1,1)} = E\left[||\hat\Sigma^{(1,1)} - \Sigma^{(1,1)}||^2_F\right] &= \sum_{i,j}E\left[(\hat\sigma_{ij} - \sigma_{ij})^2\right] \\
&= \sum_{i = j}E\left[\left((1-\lambda_1)s_{ij} + \lambda_1{s_{ij}} - \sigma_{ij}\right)^2\right] + \sum_{i \neq j}E\left[\left((1-\lambda_1)s_{ij} - \sigma_{ij}\right)^2\right] \\
&= \sum_{i = j}E\left[\left(s_{ij} -\sigma_{ij}\right)^2\right] 
 + \sum_{i \neq j}E\left[\left((1-\lambda_1)s_{ij} - \sigma_{ij}\right)^2\right] \\
\end{aligned}
\]

Next, we use the standard identity:

\[
\begin{aligned}
Var[U] &= E[U^2] - (E[U])^2 \\
E[U^2] &= Var[U] + (E[U])^2
\end{aligned}
\]

First, when \(i \neq j\), let \(U = s_{ij} - \sigma_{ij}\). Then we have

\[
\begin{aligned}
R^{(1,1)} &= \sum_{i = j}Var[s_{ij} -\sigma_{ij}] + (E[s_{ij} - \sigma_{ij}])^2
 +  \sum_{i \neq j}E\left[\left((1-\lambda_1)s_{ij} - \sigma_{ij}\right)^2\right] \\
&=  \sum_{i = j}Var[s_{ij}]  +  \sum_{i \neq j}E\left[\left((1-\lambda_1)s_{ij} - \sigma_{ij}\right)^2\right] 
\end{aligned}
\] where the latter line arises because \(s_{ij}\) is an unbiased
estimator of \(\sigma_{ij}\). Next, we consider the term when
\(i \neq j\), letting \(U = (1-\lambda_1)s_{ij} - \sigma_{ij}\).

\[
\begin{aligned}
R^{(1,1)} &=  \sum_{i = j}Var[s_{ij}]  +  \sum_{i \neq j} 
Var[(1-\lambda_1)s_{ij} - \sigma_{ij}] + (E[(1-\lambda_1)s_{ij} - \sigma_{ij}])^2 \\
&= \sum_{i=j}Var[s_{ij}] + \sum_{i \neq j}(1-\lambda_1)^2Var[s_{ij}] + ((1-\lambda_1)E[s_{ij}] - \sigma_{ij})^2 \\
&= \sum_{i=j}Var[s_{ij}] + \sum_{i \neq j}(1-\lambda_1)^2Var[s_{ij}] + \lambda_1^2(E[s_{ij}])^2  \\
&= \sum_{i=j} Var[s_{ij}] + \sum_{i \neq j}Var[s_{ij}]-2\lambda_1Var[s_{ij}] + \lambda_1^2Var[s_{ij}] + \lambda_1^2(E[s_{ij}])^2 \\
&= \sum_{i,j}Var[s_{ij}] + \sum_{i \neq j}-2\lambda_1Var[s_{ij}] + \lambda_1^2E[s_{ij}^2]
\end{aligned}
\]

\hypertarget{case-ii-both-omics-variables-in-layer-2}{%
\subsubsection{Case (ii): both omics variables in layer
2}\label{case-ii-both-omics-variables-in-layer-2}}

In this case, we can follow the analogy to the argument above to arrive
at

\[
R^{(2,2)} = \sum_{i,j}Var[s_{ij}] + \sum_{i \neq j}-2\lambda_2Var[s_{ij}] + \lambda_2^2E[s_{ij}^2]
\]

\hypertarget{case-iii-variable-i-is-in-omics-layer-1-and-variable-j-is-in-omics-layer-2.}{%
\subsubsection{\texorpdfstring{Case (iii): variable \(i\) is in omics
layer 1 and variable \(j\) is in omics layer
2.}{Case (iii): variable i is in omics layer 1 and variable j is in omics layer 2.}}\label{case-iii-variable-i-is-in-omics-layer-1-and-variable-j-is-in-omics-layer-2.}}

In this case, we derive \(R^{(1,2)}\), noting that
\(R^{(2,1)} = R^{(1,2)}\) by symmetry.

\[
\begin{aligned}
R^{(1,2)} &= E\left[||\hat\Sigma^{(1,2)} - \Sigma^{(1,2)}||^2_F\right] \\
&= \sum_{i,j}E\left[(\hat\sigma_{ij}-\sigma_{ij})^2\right] \\
&= \sum_{i,j} Var[\hat\sigma_{ij}-\sigma_{ij}] + (E[\hat\sigma_{ij}-\sigma_{ij}])^2 \\
&= \sum_{i,j} (1-\lambda_1)(1-\lambda_2)Var[s_{ij}] + (E[\hat\sigma_{ij}-s_{ij}+s_{ij}-\sigma_{ij}])^2\\
&= \sum_{i,j} (1-\lambda_1)(1-\lambda_2)Var[s_{ij}] + (E[\sqrt{1-\lambda_1}\sqrt{1-\lambda_2}s_{ij}-s_{ij}] + E[s_{ij}-\sigma_{ij}])^2 \\
&= \sum_{i,j} (1-\lambda_1)(1-\lambda_2)Var[s_{ij}] + (\sqrt{1-\lambda_1}\sqrt{1-\lambda_2}-1)^2(E[s_{ij}])^2
\end{aligned}
\]

Now the expression is in terms of the variance and expectation of the
sample covariance, which we can easily approximate with moment
estimators. We continue manipulating the expression:

\[
\begin{aligned}
R^{(1,2)} &= \sum_{i,j} \left\{1-\lambda_1-\lambda_2+\lambda_1\lambda_2\right\}Var[s_{ij}] + \left\{(1-\lambda_1)(1-\lambda_2) - 2\sqrt{1-\lambda_1}\sqrt{1-\lambda_2} + 1\right\}(E[s_{ij}])^2 \\
&= \sum_{i,j} \left\{1-\lambda_1-\lambda_2+\lambda_1\lambda_2\right\}Var[s_{ij}] + \left\{2 -\lambda_1 -\lambda_2 + \lambda_1\lambda_2 - 2\sqrt{1-\lambda_1}\sqrt{1-\lambda_2}\right\}(E[s_{ij}])^2  \\
\end{aligned}
\] Next, we again apply the identity \((E[U])^2=E[U^2]-Var[U]\):

\[
\begin{aligned}
&= \sum_{i,j} \left\{1-\lambda_1-\lambda_2+\lambda_1\lambda_2\right\}Var[s_{ij}] + \left\{2 -\lambda_1 -\lambda_2 + \lambda_1\lambda_2 - 2\sqrt{1-\lambda_1}\sqrt{1-\lambda_2}\right\}(E[s_{ij}^2]-Var[s_{ij}]) \\
& = \sum_{i,j}\left\{1-\lambda_1-\lambda_2+\lambda_1\lambda_2  - (2 - \lambda_1 - \lambda_2 + \lambda_1\lambda_2 - 2\sqrt{1-\lambda_1}\sqrt{1-\lambda_2})\right\}Var[s_{ij}] \\
& + \left\{2 - \lambda_1 - \lambda_2 + \lambda_1\lambda_2 - 2\sqrt{1-\lambda_1}\sqrt{1-\lambda_2}\right\}E[s_{ij}^2] \\
& = \sum_{i,j}\left\{-1 + 2\sqrt{1-\lambda_1}\sqrt{1-\lambda_2}\right\}Var[s_{ij}] + \left\{2 - \lambda_1 - \lambda_2 + \lambda_1\lambda_2 - 2\sqrt{1-\lambda_1}\sqrt{1-\lambda_2}\right\}E[s_{ij}^2] \\
&= -\sum_{i,j}Var[s_{ij}] + 2\sqrt{1-\lambda_1}\sqrt{1-\lambda_2}\sum_{i,j}(Var[s_{ij}] - E[s_{ij}^2]) + (2-\lambda_1 - \lambda_2 + \lambda_1\lambda_2)\sum_{i,j}E[s_{ij}^2])
\end{aligned}
\]

\hypertarget{combining-cases-i-ii-and-iii}{%
\subsubsection{Combining cases (i), (ii), and
(iii)}\label{combining-cases-i-ii-and-iii}}

Our final expression for \(R\) is

\[
\begin{aligned}
R &= R^{(1,1)} + R^{(2,2)} + 2R^{(1,2)} \\
&= \sum_{i,j}Var[s_{ij}] + \sum_{i \neq j}-2\lambda_1Var[s_{ij}] + \lambda_1^2E[s_{ij}^2] \\
& + \sum_{i,j}Var[s_{ij}] + \sum_{i \neq j}-2\lambda_2Var[s_{ij}] + \lambda_2^2E[s_{ij}^2]  \\
& + 2*\left\{-\sum_{i,j}Var[s_{ij}] + 2\sqrt{1-\lambda_1}\sqrt{1-\lambda_2}\sum_{i,j}(Var[s_{ij}] - E[s_{ij}^2]) + (2-\lambda_1 - \lambda_2 + \lambda_1\lambda_2)\sum_{i,j}E[s_{ij}^2])\right\} \\
&= \sum_{i \neq j}-2\lambda_1Var[s_{ij}] + \lambda_1^2E[s_{ij}^2]  + -2\lambda_2Var[s_{ij}] + \lambda_2^2E[s_{ij}^2] \\
& +4\sqrt{1-\lambda_1}\sqrt{1-\lambda_2} \sum_{i,j}(Var[s_{ij}] - E[s_{ij}^2])  \\
& + 2*(2-\lambda_1 - \lambda_2 + \lambda_1\lambda_2)\sum_{i,j}E[s_{ij}^2] \\
\end{aligned}
\]

Combining like terms for functions of \(\lambda_1\) and \(\lambda_2\),
we can write

\[
\begin{aligned}
R &= \sum_{i \neq j}-2\lambda_1Var[s_{ij}] + \lambda_1^2E[s_{ij}^2]  + -2\lambda_2Var[s_{ij}] + \lambda_2^2E[s_{ij}^2] \\
& + 4\sqrt{1-\lambda_1}\sqrt{1-\lambda_2} \sum_{i,j}(Var[s_{ij}] - E[s_{ij}^2])  \\
& + 2*(2-\lambda_1 - \lambda_2 + \lambda_1\lambda_2)\sum_{i,j}E[s_{ij}^2] \\
& = 4 \sum_{i,j} E[s_{ij}^2 \\
& + \lambda_1 \left\{-2\sum_{i \neq j} Var[s_{ij}] - 2\sum_{i,j}E[s_{ij}^2]\right\} \\
& + \lambda_2 \left\{-2\sum_{i \neq j} Var[s_{ij}] - 2\sum_{i,j}E[s_{ij}^2]\right\} \\
& + \lambda_1^2\sum_{i \neq j}E[s_{ij}^2] \\
& + \lambda_2^2\sum_{i \neq j}E[s_{ij}^2] \\
& + \lambda_1\lambda_2\left\{2\sum_{i \neq j}E[s_{ij}^2]\right\} \\
& + \sqrt{1-\lambda_1}\sqrt{1-\lambda_2}\left\{4\sum_{i,j}(Var[s_{ij}] - E[s_{ij}^2])\right\}
\end{aligned}
\]

This expression is of the form

\[
R = \text{const.} + \lambda_1T_1^{(1)} + \lambda_2T_1^{(2)} + \lambda_1^2T_2^{(1)} + \lambda_1^2T_2^{(2)} + \lambda_1\lambda_2T_3 + \sqrt{1-\lambda_1}\sqrt{1-\lambda_2}T_4
\]

where the first term is a constant with respect to \(\lambda_1\) and
\(\lambda_2\) and the remaining terms \(T\) are defined as:

\[
\begin{aligned}
T_1^{(k)} &= -2\left(\sum_{i \neq j} Var[s_{ij}] + \sum_{i,j}E[s_{ij}^2]\right) ; k = 1,2 \\
T_2^{(k)} &= \sum_{i \neq j}E[s_{ij}^2] ; k = 1,2 \\
T_3 &= 2\sum_{i,j}E[s_{ij}^2] \\
T_4 &= 4\sum_{i,j}(Var[s_{ij}] - E[s_{ij}^2]) \\
\end{aligned}
\] \# Remaining to do

These sums are only over a subset of the i and j, that correspond to
e.g.~omics 1 and omics 2. We need to add some notation to reflect this.

\end{document}
